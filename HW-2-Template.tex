\documentclass[10pt,twoside]{article}

\usepackage{amssymb,amsmath,amsthm,amsfonts, epsfig, graphicx, dsfont,
  bbm, bbold, url, color, setspace, multirow,pinlabel}
\usepackage[all]{xy}

\usepackage{fancyhdr} \setlength{\voffset}{-1in}
\setlength{\topmargin}{0in} \setlength{\textheight}{9.5in}
\setlength{\textwidth}{6.5in} \setlength{\hoffset}{0in}
\setlength{\oddsidemargin}{0in} \setlength{\evensidemargin}{0in}
\setlength{\marginparsep}{0in} \setlength{\marginparwidth}{0in}
\setlength{\headsep}{0.25in} \setlength{\headheight}{0.5in}
\pagestyle{fancy}

\onehalfspace

\fancyhead[LO,LE]{Math 250 - Neha Gupta} \fancyhead[RO,RE]{Relevant for Quiz 2 on 01/28}
\chead{\textbf{}} \cfoot{}
\fancyfoot[LO,LE]{} \fancyfoot[RO,RE]{Page \thepage\ of
  \pageref{LastPage}} \renewcommand{\footrulewidth}{0.5pt}
\parindent 0in
%% ------------------------------------------------------%%
%% -------------------Begin Document---------------------%%
%% ------------------------------------------------------%%
\begin{document}

\begin{center}
\huge{\bf{Homework 2} - Student Name (First and Last)}
\end{center}

\medskip

\noindent \large{\textbf{Collaborators:}}

\medskip

\begin{enumerate}
\item Read Sections 1.2, 1.3
	\item \#1.1.4: Write the negation of each of the following statements.
		\begin{itemize}
	\item[a.] All triangles are isosceles. 
	 \begin{proof}[Answer]
	    \end{proof}
	\item[b.]
	\item[c.]
	\item[e.]
	\end{itemize}
	\item 1.1: \#5(a, b, e) 
         \item 1.2: \#2(d)
	\item 1.2: \#5(a,b,c,d)
	\item 1.2: \#3 
	\item 1.2: \#D4
	\item 1.1: \#6
	\item 1.2: \#9
        \item 1.3: \#2
        \item 1.3: \#6
          \item Use truth tables to prove that each of the following statement forms are \emph{not} tautologies. These implication are common logical fallacies (errors in reasoning) since the conclusion does not follow from the hypotheses.
  		 \begin{enumerate}
  			  \item $[(P\Rightarrow Q)\wedge Q]\Rightarrow P$
  			  \item $[(P\Rightarrow Q)\wedge\neg{P}]\Rightarrow\neg{Q}$
 		\end{enumerate}
		\item Use truth tables to prove that each of the following statement forms are tautologies. These are the four most important ``rules of inference'' in propositional logic. Each rule gives a conclusion that follows from a set of hypotheses and thus give building blocks for correct proofs.
		\newpage
 		 \begin{enumerate}
   			\item $[P\wedge(P\Rightarrow Q)]\Rightarrow Q$
			\begin{proof}[Answer]
			Here is the truth table: 
			
\begin{center}
\begin{tabular}{|c|c|c|c|c|}
P & Q & P $\Rightarrow$ Q & P $\wedge$ (P $\Rightarrow$ Q)  & [P $\wedge$ (P $\Rightarrow$ Q)] $\Rightarrow$ P\\ 
\hline 
T & T & blah & bla & blaa \\
T & F & blah & bla & blaa \\
F & T  & blah & bla & blaa \\
F & F  & blah & bla & blaa \\
\end{tabular}
\end{center}
		
			\end{proof}
			\item $[\neg Q\wedge(P\Rightarrow Q)]\Rightarrow\neg{P} $
			\item $[(P\Rightarrow Q)\wedge(Q\Rightarrow R)]\Rightarrow(P\Rightarrow R)$
    			\item $[(P\vee Q)\wedge\neg{P}]\Rightarrow Q$
		\end{enumerate}

  	\end{enumerate}


\label{LastPage}
\end{document}
