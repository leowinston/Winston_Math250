\documentclass[10pt,twoside]{article}

\usepackage{amssymb,amsmath,amsthm,amsfonts, epsfig, graphicx, dsfont,
  bbm, bbold, url, color, setspace, multirow,pinlabel,multicol}
\usepackage[all]{xy}

\usepackage{fancyhdr} \setlength{\voffset}{-1in}
\setlength{\topmargin}{0in} \setlength{\textheight}{9.5in}
\setlength{\textwidth}{6.5in} \setlength{\hoffset}{0in}
\setlength{\oddsidemargin}{0in} \setlength{\evensidemargin}{0in}
\setlength{\marginparsep}{0in} \setlength{\marginparwidth}{0in}
\setlength{\headsep}{0.25in} \setlength{\headheight}{0.5in}
\pagestyle{fancy}

\newcommand{\defn}{\paragraph*{Definition}}

\onehalfspace

\fancyhead[LO,LE]{Math 250 - Neha Gupta} \fancyhead[RO,RE]{Relevant for Quiz 3 on 02/04}
\chead{\textbf{}} \cfoot{}
\fancyfoot[LO,LE]{} \fancyfoot[RO,RE]{Page \thepage\ of
  \pageref{LastPage}} \renewcommand{\footrulewidth}{0.5pt}
\parindent 0in
%% ------------------------------------------------------%%
%% -------------------Begin Document---------------------%%
%% ------------------------------------------------------%%
\begin{document}

\begin{center}
\huge{\bf{Homework 3} - Student Name (First and Last)}
\end{center}

\medskip

\noindent \large{\textbf{Collaborators:}}

\medskip

\begin{enumerate}
\item Read Sections 1.4, 2.1, 2.2 (up to Cartesian Products)
	\item Late one night in your dorm, your roommate turns to you and says:
\begin{center}
``If I don't work hard I will sleep. If I am worried I will not sleep. Thus, if I am worried I will work hard."
\end{center}
Is your friend's deduction logically correct? Justify your answer. 

\item \begin{enumerate}
\item Give an example (not used in class) of a \textit{true} conditional statement with a \textit{false} converse, or explain why no such example can exist.
\item Give an example (not used in class) of a \textit{true} conditional statement with a \textit{false} contrapositive, or explain why no such example can exist.
\item Suppose you come across the following definition:
\defn The \underline{inverse} of the conditional statement ``If P, then Q" is the conditional statement ``If not P, then not Q". 

Write the inverse of the two conditionals you gave in parts \textbf{a} and \textbf{b}. Give an example (not used in class) of a \textit{false} conditional statement with a \textit{false} inverse, or explain why no such example can exist.
\end{enumerate}


\item 1.4: \#4
 \begin{proof}[Answer]
	    \end{proof}
	    
         \item 1.4: \#8
          \item Prove that for all positive real numbers $x$, the sum of $x$
    and its reciprocal is greater than or equal to $2$.
     \begin{proof}
	    \end{proof}
    \item \label{perf} A perfect square is an integer $n$ for which there exists an
    integer $m$ such that $n=m^2$. Prove that if $n$ is a positive
    integer of the form $4k+2$ or $4k+3$ for some integer $k$, then $n$ is
    not a perfect square.
    \item Use your result for \#\ref{perf} above to prove 1.4, \#17. 
        \item Use your result for \#\ref{perf} above to prove 1.4, \#18. 

\item Prove that any real-valued solution to $x^3+x=1$ must be irrational.
\item 2.1: \#1(a, b, c, d, e)
\item For each part below, give an example, if there is one, of sets $A$, $B$, and $C$ satisfying the given statement:
%\begin{multicols}{2}
\begin{enumerate}
\item $A\subseteq B$, $B\nsubseteq C$ and $A\subseteq C$.
\item $A\subseteq B$, $B\subseteq C$ and $C\subseteq A$.
\item $A\nsubseteq B$, $B\nsubseteq C$ and $A\subseteq C$.
\item $A\subseteq B$, $B\nsubseteq C$ and $A\nsubseteq C$.
\end{enumerate}
%\end{multicols}
\item 2.1: \# 14
\item Let $A, B$, and $C$ be arbitrary sets. Prove that if $A-C \nsubseteq A-B$, then $B \nsubseteq C$.  
\item 2.2: \# 1
\item 2.2: \# 2
\item 2.2: \# 4
  	\end{enumerate}

\label{LastPage}
\end{document}
